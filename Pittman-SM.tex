% Created 2022-07-19 Tue 09:19
% Intended LaTeX compiler: pdflatex
\documentclass[10pt,leftblank,twoside]{mitthesis}
\usepackage[utf8]{inputenc}
\usepackage[T1]{fontenc}
\usepackage{graphicx}
\usepackage{longtable}
\usepackage{wrapfig}
\usepackage{rotating}
\usepackage[normalem]{ulem}
\usepackage{amsmath}
\usepackage{amssymb}
\usepackage{capt-of}
\usepackage{hyperref}
\usepackage{minted}
\usepackage{lgrind}
\usepackage{cmap}
\usepackage[T1]{fontenc}
\pagestyle{plain}
\def\all{all}
\ifx\files\all \typeout{Including all files.} \else
\date{}
\title{Main SM Thesis Cover 0 Introduction 1 Lit Review}
\hypersetup{
 pdfauthor={Cameron W Pittman},
 pdftitle={Main SM Thesis Cover 0 Introduction 1 Lit Review},
 pdfkeywords={},
 pdfsubject={},
 pdfcreator={Emacs 29.0.50 (Org mode 9.6)}, 
 pdflang={English}}
\makeatletter
\newcommand{\citeprocitem}[2]{\hyper@linkstart{cite}{citeproc_bib_item_#1}#2\hyper@linkend}
\makeatother

\usepackage[notquote]{hanging}
\begin{document}

%% The documentclass options along with the pagestyle can be used to generate
%% a technical report, a draft copy, or a regular thesis. You may need to
%% re-specify the pagestyle after you \include cover.tex. For more
%% information, see the first few lines of mitthesis.cls.
%%
%\documentclass[12pt,vi,twoside]{mitthesis}
%%
%%  If you want your thesis copyright to you instead of MIT, use the
%%  ``vi'' option, as above.
%%
%\documentclass[12pt,twoside,leftblank]{mitthesis}
%%
%% If you want blank pages before new chapters to be labelled ``This
%% Page Intentionally Left Blank'', use the ``leftblank'' option, as
%% above.

%% These have been added at the request of the MIT Libraries, because
%% some PDF conversions mess up the ligatures.  -LB, 1/22/2014
%% This bit allows you to either specify only the files which you wish to
%% process, or `all' to process all files which you \include.
%% Krishna Sethuraman (1990).

%\typein [\files]{Enter file names to process, (chap1,chap2 ...), or `all' to process all files:}
%\typeout{Including only \files.} \includeonly{\files} \fi

\hypersetup{
  pdfauthor={Cameron W. Pittman},
  pdftitle={Distributed Multi-Agent Decision Making Under Uncertain Communication},
  pdfkeywords={},
  pdfsubject={},
  pdfcreator={Emacs 29.0.50 (Org mode 9.6)},
  pdflang={English}}

% NOTE:
% These templates make an effort to conform to the MIT Thesis specifications,
% however the specifications can change. We recommend that you verify the
% layout of your title page with your thesis advisor and/or the MIT
% Libraries before printing your final copy.

% FYI, the title and thesis need to be defined in the \hypersetup section in the main org file
\title{Distributed Multi-Agent Decision Making Under Uncertain Communication}
\author{Cameron W. Pittman}

% If you wish to list your previous degrees on the cover page, use the
% previous degrees command:
% You can use the \\ command to list multiple previous degrees
\prevdegrees{M.A., Belmont University (2011) \\
    B.A., Vanderbilt University (2009)}

\department{Department of Aeronautics and Astronautics}

% If the thesis is for two degrees simultaneously, list them both
% separated by \and like this:
% \degree{Doctor of Philosophy \and Master of Science}
\degree{Master of Science}

% As of the 2007-08 academic year, valid degree months are September,
% February, or June.  The default is June.
\degreemonth{June}
\degreeyear{2023}
\thesisdate{May 1, 2023}

%% By default, the thesis will be copyrighted to MIT.  If you need to copyright
%% the thesis to yourself, just specify the `vi' documentclass option.  If for
%% some reason you want to exactly specify the copyright notice text, you can
%% use the \copyrightnoticetext command.
%\copyrightnoticetext{\copyright IBM, 1990.  Do not open till Xmas.}

% If there is more than one supervisor, use the \supervisor command
% once for each.
\supervisor{Brian C. Williams}{Professor of Aeronautics and Astronautics, MIT}

% This is the department committee chairman, not the thesis committee
% chairman.  You should replace this with your Department's Committee
% Chairman.
\chairman{Jonathan How}{R.C Maclaurin Professor of Aeronautics and Astronautics, MIT \\
    Chair, Graduate Program Committee}

% Make the titlepage based on the above information.  If you need
% something special and can't use the standard form, you can specify
% the exact text of the titlepage yourself.  Put it in a titlepage
% environment and leave blank lines where you want vertical space.
% The spaces will be adjusted to fill the entire page.  The dotted
% lines for the signatures are made with the \signature command.
\maketitle


% The abstractpage environment sets up everything on the page except
% the text itself.  The title and other header material are put at the
% top of the page, and the supervisors are listed at the bottom.  A
% new page is begun both before and after.  Of course, an abstract may
% be more than one page itself.  If you need more control over the
% format of the page, you can use the abstract environment, which puts
% the word "Abstract" at the beginning and single spaces its text.

%% You can either \input (*not* \include) your abstract file, or you can put
%% the text of the abstract directly between the \begin{abstractpage} and
%% \end{abstractpage} commands.

% First copy: start a new page, and save the page number.
\cleardoublepage
% Uncomment the next line if you do NOT want a page number on your
% abstract and acknowledgments pages.
% \pagestyle{empty}
\setcounter{savepage}{\thepage}
\begin{abstractpage}

Lorem ipsum dolor sit amet, consectetur adipiscing elit. Nam quis neque et erat laoreet finibus at
ac leo. Curabitur pellentesque, diam quis dignissim finibus, enim dui feugiat leo, nec porttitor
sapien mi ac felis. Nam aliquam pretium nibh, quis dapibus dolor gravida sit amet. Cras porttitor
dui quis elementum pulvinar. Nulla id pulvinar massa. Nullam ut diam non lorem venenatis faucibus.
Vivamus lacus ante, pellentesque vitae nisl sit amet, bibendum facilisis purus.

\end{abstractpage}

\cleardoublepage

\section*{Acknowledgements}

So long and thanks for all the fish!

\pagestyle{plain}

\tableofcontents
\newpage
\listoffigures
\newpage
\listoftables

\begin{document}

\chapter{Introduction}
\label{sec:orge988725}

This is the introduction.

\section{Section Header}
\label{sec:orgb0e7f60}

This is a section of text. As said in\ldots{} \citeprocitem{1}{[1]} ``Hey''. So said \citeprocitem{2}{[2]} too
that things are cool.

\chapter{Literature Review}
\label{sec:org4db06a5}

More!

\section{Section 2}
\label{sec:org8111d31}

And more! Wow, said \citeprocitem{3}{[3]}. That's neat.

\appendix

%% This defines the bibliography file (main.bib) and the bibliography style.
%% If you want to create a bibliography file by hand, change the contents of
%% this file to a `thebibliography' environment.  For more information
%% see section 4.3 of the LaTeX manual.
% \bibliography{library}
\begin{singlespace}
\begin{thebibliography}

\begin{hangparas}{1.5em}{1}
\hypertarget{citeproc_bib_item_1}{[1] N. Bhargava, C. Muise, and B. C. Williams, “Variable-delay controllability,” \textit{Ijcai international joint conference on artificial intelligence}, vol. 2018-July, pp. 4660–4666, 2018, doi: \href{https://doi.org/10.24963/ijcai.2018/648}{10.24963/ijcai.2018/648}.\\}

\hypertarget{citeproc_bib_item_2}{[2] M. J. Miller, “Decision Support System Development For Human Extravehicular Activity,” Georgia Institute of Technology, 2017.\\}

\hypertarget{citeproc_bib_item_3}{[3] B. C. Williams, M. D. Ingham, S. H. Chung, and P. H. Elliott, “Model-based programming of intelligent embedded systems and robotic space explorers,” \textit{Proceedings of the ieee}, vol. 91, no. 1, pp. 212–236, 2003, doi: \href{https://doi.org/10.1109/JPROC.2002.805828}{10.1109/JPROC.2002.805828}.\\}
\end{hangparas}

\end{thebibliography}
\end{singlespace}
\end{document}
\end{document}